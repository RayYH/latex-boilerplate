% colors
\definecolor{dlighgray}{gray}{0.9}
\definecolor{dblue}{RGB}{102, 120, 173}
\definecolor{ddblue}{rgb}{0.06, 0.75, 0.99}
\definecolor{dgreen}{RGB}{118, 167, 125}
\definecolor{dred}{RGB}{198, 113, 113}
\definecolor{dorange}{RGB}{230, 169, 132}
\definecolor{dtan}{RGB}{221, 215, 200}
\definecolor{dgray}{RGB}{94, 94, 94}
\definecolor{ddgray}{RGB}{46, 49, 49}
\definecolor{dnote}{HTML}{0288d1}
\definecolor{dhint}{HTML}{009688}
\definecolor{ddanger}{HTML}{c2185b}
\definecolor{dcaution}{HTML}{ffa726}
\definecolor{derror}{HTML}{d32f2f}
\definecolor{dattention}{HTML}{455a64}
% inline code
\newcommand{\dic}{\mintinline}
% emphasize
\newcommand{\dem}[1]{\textbf{\textcolor{ddblue}{#1}}}
% float right
\newcommand{\dfr}[1]{\null\hfill {#1}}
% admonitions
\newcommand{\danger}[2]{
    \begin{tcolorbox}[enlarge top initially by=3mm,colback=ddanger!5!white,colframe=ddanger!75!black,title={\textbf{#1}}]{#2}
    \end{tcolorbox}
}
\newcommand{\note}[2]{
    \begin{tcolorbox}[enlarge top initially by=3mm,colback=dnote!5!white,colframe=dnote!75!black,title={\textbf{#1}}]{#2}
    \end{tcolorbox}
}
\newcommand{\hint}[2]{
    \begin{tcolorbox}[enlarge top initially by=3mm,colback=dhint!5!white,colframe=dhint!75!black,title={\textbf{#1}}]{#2}
    \end{tcolorbox}
}
\newcommand{\caution}[2]{
    \begin{tcolorbox}[enlarge top initially by=3mm,colback=dcaution!5!white,colframe=dcaution!75!black,title={\textbf{#1}}]{#2}
    \end{tcolorbox}
}
\newcommand{\error}[2]{
    \begin{tcolorbox}[enlarge top initially by=3mm,colback=derror!5!white,colframe=derror!75!black,title={\textbf{#1}}]{#2}
    \end{tcolorbox}
}
\newcommand{\attention}[2]{
    \begin{tcolorbox}[enlarge top initially by=3mm,colback=dattention!5!white,colframe=dattention!75!black,title={\textbf{#1}}]{#2}
    \end{tcolorbox}
}
% tikz libraries
\usetikzlibrary{shadows.blur}
\usetikzlibrary{shapes.symbols}
% Put a fancy box around things.
\newcommand{\dbox}[1]{
    \begin{mdframed}[roundcorner=4pt, backgroundcolor=gray!5]
        \vspace{1mm}
        {#1}
    \end{mdframed}
}
% keyboard
\newcommand*\keystroke[1]{%
    \tikz[baseline=(key.base)]
    \node[
        draw,
        fill=white,
        drop shadow={shadow xshift=0.25ex,shadow yshift=-0.25ex,fill=black,opacity=0.75},
        rectangle,
        rounded corners=2pt,
        inner sep=1pt,
        line width=0.5pt,
        font=\scriptsize\sffamily
    ](key) {#1\strut}
    ;
}
% enable 1.1.1.1 section style
\newcommand{\dparagraph}[1]{

    \paragraph{#1}\mbox{}\\}
% URL
\newcommand{\durl}[1]{\textcolor{dblue}{\underline{\url{#1}}}}
% Circled Numbers
% http://tex.stackexchange.com/questions/7032/good-way-to-make-textcircled-numbers
\newcommand*\circled[1]{\tikz[baseline=(char.base)]{\node[shape=circle,draw,inner sep=0.7pt] (char) {\footnotesize{#1}};}}
% Dynamically sized mid bar
\newcommand{\bigmid}{\mathrel{\Big|}}
% text colors
\newcommand{\tc}[2]{\textcolor{#1}{#2}}
\newcommand{\spacerule}{\begin{center}
        \hdashrule{2cm}{1pt}{1pt}
    \end{center}}
% full width image
\newcommand{\image}[2]{
    \begin{figure}[h]
        \includegraphics[width=1\textwidth]{#1}
        \caption{#2}
        \label{fig:#1}
    \end{figure}}
% scale with caption
\newcommand{\simage}[3]{
    \begin{figure}[h]
        \centering
        \includegraphics[scale=#3]{#1}
        \caption{#2}
        \label{fig:#1}
    \end{figure}
}
% img
\newcommand{\img}[2]{
    \begin{center}
        \includegraphics[scale=#2]{#1}
    \end{center}
}
% minus sign
\newcommand{\minus}{\scalebox{0.75}[1.0]{$-$}}
% wrap any text as a figure
\newcommand{\wrapfigure}[3]{
    \begin{figure}[h]
        {#3}
        \caption{#2}
        \label{fig:#1}
    \end{figure}
}
